% Options for packages loaded elsewhere
\PassOptionsToPackage{unicode}{hyperref}
\PassOptionsToPackage{hyphens}{url}
%
\documentclass[
]{article}
\usepackage{lmodern}
\usepackage{amssymb,amsmath}
\usepackage{ifxetex,ifluatex}
\ifnum 0\ifxetex 1\fi\ifluatex 1\fi=0 % if pdftex
  \usepackage[T1]{fontenc}
  \usepackage[utf8]{inputenc}
  \usepackage{textcomp} % provide euro and other symbols
\else % if luatex or xetex
  \usepackage{unicode-math}
  \defaultfontfeatures{Scale=MatchLowercase}
  \defaultfontfeatures[\rmfamily]{Ligatures=TeX,Scale=1}
\fi
% Use upquote if available, for straight quotes in verbatim environments
\IfFileExists{upquote.sty}{\usepackage{upquote}}{}
\IfFileExists{microtype.sty}{% use microtype if available
  \usepackage[]{microtype}
  \UseMicrotypeSet[protrusion]{basicmath} % disable protrusion for tt fonts
}{}
\makeatletter
\@ifundefined{KOMAClassName}{% if non-KOMA class
  \IfFileExists{parskip.sty}{%
    \usepackage{parskip}
  }{% else
    \setlength{\parindent}{0pt}
    \setlength{\parskip}{6pt plus 2pt minus 1pt}}
}{% if KOMA class
  \KOMAoptions{parskip=half}}
\makeatother
\usepackage{xcolor}
\IfFileExists{xurl.sty}{\usepackage{xurl}}{} % add URL line breaks if available
\IfFileExists{bookmark.sty}{\usepackage{bookmark}}{\usepackage{hyperref}}
\hypersetup{
  pdftitle={R Notebook},
  hidelinks,
  pdfcreator={LaTeX via pandoc}}
\urlstyle{same} % disable monospaced font for URLs
\usepackage[margin=1in]{geometry}
\usepackage{color}
\usepackage{fancyvrb}
\newcommand{\VerbBar}{|}
\newcommand{\VERB}{\Verb[commandchars=\\\{\}]}
\DefineVerbatimEnvironment{Highlighting}{Verbatim}{commandchars=\\\{\}}
% Add ',fontsize=\small' for more characters per line
\usepackage{framed}
\definecolor{shadecolor}{RGB}{248,248,248}
\newenvironment{Shaded}{\begin{snugshade}}{\end{snugshade}}
\newcommand{\AlertTok}[1]{\textcolor[rgb]{0.94,0.16,0.16}{#1}}
\newcommand{\AnnotationTok}[1]{\textcolor[rgb]{0.56,0.35,0.01}{\textbf{\textit{#1}}}}
\newcommand{\AttributeTok}[1]{\textcolor[rgb]{0.77,0.63,0.00}{#1}}
\newcommand{\BaseNTok}[1]{\textcolor[rgb]{0.00,0.00,0.81}{#1}}
\newcommand{\BuiltInTok}[1]{#1}
\newcommand{\CharTok}[1]{\textcolor[rgb]{0.31,0.60,0.02}{#1}}
\newcommand{\CommentTok}[1]{\textcolor[rgb]{0.56,0.35,0.01}{\textit{#1}}}
\newcommand{\CommentVarTok}[1]{\textcolor[rgb]{0.56,0.35,0.01}{\textbf{\textit{#1}}}}
\newcommand{\ConstantTok}[1]{\textcolor[rgb]{0.00,0.00,0.00}{#1}}
\newcommand{\ControlFlowTok}[1]{\textcolor[rgb]{0.13,0.29,0.53}{\textbf{#1}}}
\newcommand{\DataTypeTok}[1]{\textcolor[rgb]{0.13,0.29,0.53}{#1}}
\newcommand{\DecValTok}[1]{\textcolor[rgb]{0.00,0.00,0.81}{#1}}
\newcommand{\DocumentationTok}[1]{\textcolor[rgb]{0.56,0.35,0.01}{\textbf{\textit{#1}}}}
\newcommand{\ErrorTok}[1]{\textcolor[rgb]{0.64,0.00,0.00}{\textbf{#1}}}
\newcommand{\ExtensionTok}[1]{#1}
\newcommand{\FloatTok}[1]{\textcolor[rgb]{0.00,0.00,0.81}{#1}}
\newcommand{\FunctionTok}[1]{\textcolor[rgb]{0.00,0.00,0.00}{#1}}
\newcommand{\ImportTok}[1]{#1}
\newcommand{\InformationTok}[1]{\textcolor[rgb]{0.56,0.35,0.01}{\textbf{\textit{#1}}}}
\newcommand{\KeywordTok}[1]{\textcolor[rgb]{0.13,0.29,0.53}{\textbf{#1}}}
\newcommand{\NormalTok}[1]{#1}
\newcommand{\OperatorTok}[1]{\textcolor[rgb]{0.81,0.36,0.00}{\textbf{#1}}}
\newcommand{\OtherTok}[1]{\textcolor[rgb]{0.56,0.35,0.01}{#1}}
\newcommand{\PreprocessorTok}[1]{\textcolor[rgb]{0.56,0.35,0.01}{\textit{#1}}}
\newcommand{\RegionMarkerTok}[1]{#1}
\newcommand{\SpecialCharTok}[1]{\textcolor[rgb]{0.00,0.00,0.00}{#1}}
\newcommand{\SpecialStringTok}[1]{\textcolor[rgb]{0.31,0.60,0.02}{#1}}
\newcommand{\StringTok}[1]{\textcolor[rgb]{0.31,0.60,0.02}{#1}}
\newcommand{\VariableTok}[1]{\textcolor[rgb]{0.00,0.00,0.00}{#1}}
\newcommand{\VerbatimStringTok}[1]{\textcolor[rgb]{0.31,0.60,0.02}{#1}}
\newcommand{\WarningTok}[1]{\textcolor[rgb]{0.56,0.35,0.01}{\textbf{\textit{#1}}}}
\usepackage{graphicx,grffile}
\makeatletter
\def\maxwidth{\ifdim\Gin@nat@width>\linewidth\linewidth\else\Gin@nat@width\fi}
\def\maxheight{\ifdim\Gin@nat@height>\textheight\textheight\else\Gin@nat@height\fi}
\makeatother
% Scale images if necessary, so that they will not overflow the page
% margins by default, and it is still possible to overwrite the defaults
% using explicit options in \includegraphics[width, height, ...]{}
\setkeys{Gin}{width=\maxwidth,height=\maxheight,keepaspectratio}
% Set default figure placement to htbp
\makeatletter
\def\fps@figure{htbp}
\makeatother
\setlength{\emergencystretch}{3em} % prevent overfull lines
\providecommand{\tightlist}{%
  \setlength{\itemsep}{0pt}\setlength{\parskip}{0pt}}
\setcounter{secnumdepth}{-\maxdimen} % remove section numbering

\title{R Notebook}
\author{}
\date{\vspace{-2.5em}}

\begin{document}
\maketitle

\hypertarget{prova-de-especializauxe7uxe3o}{%
\section{Prova de Especialização}\label{prova-de-especializauxe7uxe3o}}

\begin{itemize}
\tightlist
\item
  Primeiramente vamos importar os dados:
\end{itemize}

\begin{Shaded}
\begin{Highlighting}[]
\NormalTok{wine.data <-}\StringTok{ }\NormalTok{readxl}\OperatorTok{::}\KeywordTok{read_excel}\NormalTok{(}\StringTok{'exe.xlsx'}\NormalTok{, }\StringTok{'D'}\NormalTok{)}
\KeywordTok{head}\NormalTok{(wine.data)}
\end{Highlighting}
\end{Shaded}

\begin{verbatim}
## # A tibble: 6 x 6
##   Clari Aroma Corpo Sabor Afina   Qua
##   <dbl> <dbl> <dbl> <dbl> <dbl> <dbl>
## 1     1   3.3   2.8   3.1   4.1   9.8
## 2     1   4.4   4.9   3.5   3.9  12.6
## 3     1   3.9   5.3   4.8   4.7  11.9
## 4     1   3.9   2.6   3.1   3.6  11.1
## 5     1   5.6   5.1   5.5   5.1  13.3
## 6     1   4.6   4.7   5     4.1  12.8
\end{verbatim}

Os dados do vinho são referentes aos atributos Claridade, Aroma, Corpo,
Sabor, Afinação e Qualidade. O nosso objetivo no estudo é relacionar os
atributos Claridades, Aroma, Corpo, Sabor e Afinação com a Qualidade do
vinho, ou seja, quais desses atributos interferem na qualidade.

\hypertarget{a-estime-beta-pelo-muxe9todo-dos-muxednimos-quadrados.-explique-o-procedimento.}{%
\paragraph{\texorpdfstring{\textbf{a) Estime \(\beta\) pelo método dos
mínimos quadrados. Explique o
procedimento.}}{a) Estime \textbackslash beta pelo método dos mínimos quadrados. Explique o procedimento.}}\label{a-estime-beta-pelo-muxe9todo-dos-muxednimos-quadrados.-explique-o-procedimento.}}

O objetivo é saber os valores \(\widehat{\beta1}\),
\(\widehat{\beta2}\), \(\widehat{\beta3}\), \(\widehat{\beta4}\),
\(\widehat{\beta5}\) que minimize a soma\ldots{}

Para isso, precisa-se calcular a \emph{Qualidade Estimada}
(\(\widehat{Y}\)), que é:

\[
\widehat{Y} =  \beta0 + \beta1*Claridade + \beta2*Aroma + \beta3*Corpo + \beta4*Sabor + \beta5*Afinacao
\]

\begin{Shaded}
\begin{Highlighting}[]
\NormalTok{Claridade <-}\StringTok{ }\NormalTok{wine.data}\OperatorTok{$}\NormalTok{Clari}
\NormalTok{Aroma <-}\StringTok{ }\NormalTok{wine.data}\OperatorTok{$}\NormalTok{Aroma}
\NormalTok{Corpo <-}\StringTok{ }\NormalTok{wine.data}\OperatorTok{$}\NormalTok{Corpo}
\NormalTok{Sabor <-}\StringTok{ }\NormalTok{wine.data}\OperatorTok{$}\NormalTok{Sabor}
\NormalTok{Afinacao <-}\StringTok{ }\NormalTok{wine.data}\OperatorTok{$}\NormalTok{Afina}
\NormalTok{Y <-}\StringTok{ }\NormalTok{wine.data}\OperatorTok{$}\NormalTok{Qua}

\NormalTok{desvioY <-}\StringTok{ }\ControlFlowTok{function}\NormalTok{(betas) \{}
\NormalTok{  Y.Estimado <-}\StringTok{ }\NormalTok{betas[}\DecValTok{1}\NormalTok{] }\OperatorTok{+}\StringTok{ }\NormalTok{betas[}\DecValTok{2}\NormalTok{]}\OperatorTok{*}\NormalTok{Claridade }\OperatorTok{+}\StringTok{ }\NormalTok{betas[}\DecValTok{3}\NormalTok{]}\OperatorTok{*}\NormalTok{Aroma }\OperatorTok{+}\StringTok{ }\NormalTok{betas[}\DecValTok{4}\NormalTok{]}\OperatorTok{*}\NormalTok{Corpo }\OperatorTok{+}\StringTok{ }\NormalTok{betas[}\DecValTok{5}\NormalTok{]}\OperatorTok{*}\NormalTok{Sabor }\OperatorTok{+}\StringTok{ }\NormalTok{betas[}\DecValTok{6}\NormalTok{]}\OperatorTok{*}\NormalTok{Afinacao}
\NormalTok{  S <-}\StringTok{ }\KeywordTok{sum}\NormalTok{( (Y }\OperatorTok{-}\StringTok{ }\NormalTok{Y.Estimado)}\OperatorTok{^}\DecValTok{2}\NormalTok{ )}
\NormalTok{\}}
\end{Highlighting}
\end{Shaded}

A função \texttt{desvioY}, calcula o resíduo dos Y
(\(Qualidade Real - Qualidade Estimada\)), eleva ao quadrado para não
haver números negativos eliminando números positivos e por fim soma
tudo. É essa função que se gostaria que fosse a menor possível.

Ou seja, o objetivo é saber os valores dos \(\widehat{\beta1}\),
\(\widehat{\beta2}\), \(\widehat{\beta3}\), \(\widehat{\beta4}\),
\(\widehat{\beta5}\) que minimize a soma S.

Aplicando a função \texttt{optim} temos os betas procurados:

\begin{Shaded}
\begin{Highlighting}[]
\NormalTok{R}\FloatTok{.2}\NormalTok{ <-}\StringTok{ }\KeywordTok{optim}\NormalTok{(}\DataTypeTok{par =} \KeywordTok{c}\NormalTok{(}\DecValTok{1}\NormalTok{,}\DecValTok{1}\NormalTok{,}\DecValTok{1}\NormalTok{,}\DecValTok{1}\NormalTok{,}\DecValTok{1}\NormalTok{,}\DecValTok{1}\NormalTok{), }\DataTypeTok{fn =}\NormalTok{ desvioY, }\DataTypeTok{method =} \StringTok{"L-BFGS-B"}\NormalTok{)}
\NormalTok{R}\FloatTok{.2}\OperatorTok{$}\NormalTok{par}
\end{Highlighting}
\end{Shaded}

\begin{verbatim}
## [1]  3.9962004  2.3384721  0.4826345  0.2732280  1.1682793 -0.6837678
\end{verbatim}

Executando a regressão linear para ter valores matemáticos mais exatos,
temos:

\begin{Shaded}
\begin{Highlighting}[]
\NormalTok{U <-}\StringTok{ }\KeywordTok{lm}\NormalTok{(Y}\OperatorTok{~}\NormalTok{Claridade}\OperatorTok{+}\NormalTok{Aroma}\OperatorTok{+}\NormalTok{Corpo}\OperatorTok{+}\NormalTok{Sabor}\OperatorTok{+}\NormalTok{Afinacao)}
\NormalTok{U}\OperatorTok{$}\NormalTok{coefficients}
\end{Highlighting}
\end{Shaded}

\begin{verbatim}
## (Intercept)   Claridade       Aroma       Corpo       Sabor    Afinacao 
##   3.9968648   2.3394535   0.4825505   0.2731612   1.1683238  -0.6840102
\end{verbatim}

Comparando os valores dos \(\widehat{betas}\) pelo método dos mínimos
quadrados com os valores reais da regressão linear, observamos que os
valores estão bem próximos.

\hypertarget{b-vocuxea-concorda-que-uma-relauxe7uxe3o-linear-uxe9-adequada-como-avaliaria-no-caso-da-regressuxe3o-linear-muxfaltipla}{%
\paragraph{\texorpdfstring{\textbf{b) Você concorda que uma relação
linear é adequada? Como avaliaria no caso da Regressão Linear
Múltipla?}}{b) Você concorda que uma relação linear é adequada? Como avaliaria no caso da Regressão Linear Múltipla?}}\label{b-vocuxea-concorda-que-uma-relauxe7uxe3o-linear-uxe9-adequada-como-avaliaria-no-caso-da-regressuxe3o-linear-muxfaltipla}}

Caso tivéssemos a variável de resposta Y relacionada com uma única
variável independente X, seria suficiente fazer um diagrama de dispersão
do relacionando Y com X. Porém, nós temos 5 variáveis (dimensões), sem
adicionar o Y.

Portanto, precisamos fazer um gráfico de dispersão do resíduo com cada
uma das variáveis independentes (Claridade, Aroma, Corpo, Sabor,
Afinacao), esperando um padrão aleatório em cada um desses gráficos.

Lembrando que \emph{Resídio} é tudo que não foi explicado da qualidade
do vinho.

Plotando os gráficos relacionando as variávies independentes (Claridade,
Aroma, Corpo, Sabor, Afinacao) em relação com o Resíduo \emph{não} se
observa um padrão linear em nenhuma delas.

Logo, é razoável se trabalhar com uma Regressão Linear, pois o que se
sobrou no resíduo não possui efeito não linear no modelo.

\begin{Shaded}
\begin{Highlighting}[]
\KeywordTok{par}\NormalTok{(}\DataTypeTok{mfrow=}\KeywordTok{c}\NormalTok{(}\DecValTok{3}\NormalTok{,}\DecValTok{3}\NormalTok{))}

\KeywordTok{plot}\NormalTok{(Claridade, U}\OperatorTok{$}\NormalTok{residuals)}
\KeywordTok{plot}\NormalTok{(Aroma, U}\OperatorTok{$}\NormalTok{residuals)}
\KeywordTok{plot}\NormalTok{(Corpo, U}\OperatorTok{$}\NormalTok{residuals)}
\KeywordTok{plot}\NormalTok{(Sabor, U}\OperatorTok{$}\NormalTok{residuals)}
\KeywordTok{plot}\NormalTok{(Afinacao, U}\OperatorTok{$}\NormalTok{residuals)}
\end{Highlighting}
\end{Shaded}

\includegraphics{prova_files/figure-latex/unnamed-chunk-4-1.pdf}
\#\#\#\# \textbf{c)Estime a variância da componente erro utilizando um
estimador não viciado (\(\sigma^2\)). Explique o procedimento}

A variância não viciada do componente erro trata-se da variância dos
resíduos corrigida pelo número de parâmetro estimados, logo:

\$\sigma\^{}2 = Var(residuos) / (n - Quantidade De Parâmetros Estimados)
\$

\emph{Observação: utilizamos do artifício matemático de multiplicar por
(n -1) para eliminar o denominador da fórmula utilizada no cálculo da
variância dos resíduos.}

\begin{Shaded}
\begin{Highlighting}[]
\NormalTok{n <-}\StringTok{ }\DecValTok{38}
\NormalTok{quantidadeParametrosEstimados <-}\StringTok{ }\DecValTok{6}
\NormalTok{sigma}\FloatTok{.2}\NormalTok{ <-}\StringTok{ }\KeywordTok{var}\NormalTok{(U}\OperatorTok{$}\NormalTok{residuals) }\OperatorTok{*}\StringTok{ }\NormalTok{(n }\DecValTok{-1}\NormalTok{ ) }\OperatorTok{/}\StringTok{ }\NormalTok{(n }\OperatorTok{-}\StringTok{ }\NormalTok{quantidadeParametrosEstimados)}
\NormalTok{sigma}\FloatTok{.2}
\end{Highlighting}
\end{Shaded}

\begin{verbatim}
## [1] 1.3515
\end{verbatim}

\hypertarget{d-estime-a-variuxe2ncia-de-widehatbeta_4-relativo-ao-sabor.-explique-o-procedimento.}{%
\paragraph{\texorpdfstring{\textbf{d) Estime a variância de
\(\widehat{\beta_4}\) (relativo ao sabor). Explique o
procedimento.}}{d) Estime a variância de \textbackslash widehat\{\textbackslash beta\_4\} (relativo ao sabor). Explique o procedimento.}}\label{d-estime-a-variuxe2ncia-de-widehatbeta_4-relativo-ao-sabor.-explique-o-procedimento.}}

Iremos estimar a variância através da matriz utilizando a estimativa de
mínimos quadrados de \(\beta = (X'X)^-1\)

\begin{itemize}
\tightlist
\item
  Para obtermos a Matriz de planejamento X:
\end{itemize}

\begin{Shaded}
\begin{Highlighting}[]
\NormalTok{X <-}\StringTok{ }\KeywordTok{model.matrix}\NormalTok{(U)}
\KeywordTok{head.matrix}\NormalTok{(X)}
\end{Highlighting}
\end{Shaded}

\begin{verbatim}
##   (Intercept) Claridade Aroma Corpo Sabor Afinacao
## 1           1         1   3.3   2.8   3.1      4.1
## 2           1         1   4.4   4.9   3.5      3.9
## 3           1         1   3.9   5.3   4.8      4.7
## 4           1         1   3.9   2.6   3.1      3.6
## 5           1         1   5.6   5.1   5.5      5.1
## 6           1         1   4.6   4.7   5.0      4.1
\end{verbatim}

\begin{itemize}
\tightlist
\item
  Calculando a transposta da matriz:
\end{itemize}

\begin{Shaded}
\begin{Highlighting}[]
\NormalTok{X.linha <-}\StringTok{ }\KeywordTok{t}\NormalTok{(X)}
\KeywordTok{head.matrix}\NormalTok{(X.linha)}
\end{Highlighting}
\end{Shaded}

\begin{verbatim}
##               1   2   3   4   5   6   7   8   9  10  11  12  13  14  15  16  17
## (Intercept) 1.0 1.0 1.0 1.0 1.0 1.0 1.0 1.0 1.0 1.0 1.0 1.0 1.0 1.0 1.0 1.0 1.0
## Claridade   1.0 1.0 1.0 1.0 1.0 1.0 1.0 1.0 1.0 1.0 1.0 0.5 0.8 0.7 1.0 0.9 1.0
## Aroma       3.3 4.4 3.9 3.9 5.6 4.6 4.8 5.3 4.3 4.3 5.1 3.3 5.9 7.7 7.1 5.5 6.3
## Corpo       2.8 4.9 5.3 2.6 5.1 4.7 4.8 4.5 4.3 3.9 4.3 5.4 5.7 6.6 4.4 5.6 5.4
## Sabor       3.1 3.5 4.8 3.1 5.5 5.0 4.8 4.3 3.9 4.7 4.5 4.3 7.0 6.7 5.8 5.6 4.8
## Afinacao    4.1 3.9 4.7 3.6 5.1 4.1 3.3 5.2 2.9 3.9 3.6 3.6 4.1 3.7 4.1 4.4 4.6
##              18  19  20  21  22  23  24  25  26  27  28  29  30  31  32  33  34
## (Intercept) 1.0 1.0 1.0 1.0 1.0 1.0 1.0 1.0 1.0 1.0 1.0 1.0 1.0 1.0 1.0 1.0 1.0
## Claridade   1.0 1.0 0.9 0.9 1.0 0.7 0.7 1.0 1.0 1.0 1.0 1.0 1.0 1.0 0.8 1.0 1.0
## Aroma       5.0 4.6 3.4 6.4 5.5 4.7 4.1 6.0 4.3 3.9 5.1 3.9 4.5 5.2 4.2 3.3 6.8
## Corpo       5.5 4.1 5.0 5.4 5.3 4.1 4.0 5.4 4.6 4.0 4.9 4.4 3.7 4.3 3.8 3.5 5.0
## Sabor       5.5 4.3 3.4 6.6 5.3 5.0 4.1 5.7 4.7 5.1 5.0 5.0 2.9 5.0 3.0 4.3 6.0
## Afinacao    4.1 3.1 3.4 4.8 3.8 3.7 4.0 4.7 4.9 5.1 5.1 4.4 3.9 6.0 4.7 4.5 5.2
##              35  36  37  38
## (Intercept) 1.0 1.0 1.0 1.0
## Claridade   0.8 0.8 0.8 0.8
## Aroma       5.0 3.5 4.3 5.2
## Corpo       5.7 4.7 5.5 4.8
## Sabor       5.5 4.2 3.5 5.7
## Afinacao    4.8 3.3 5.8 3.5
\end{verbatim}

\begin{itemize}
\tightlist
\item
  Multiplica a tranposta (X') por X:
\end{itemize}

\begin{Shaded}
\begin{Highlighting}[]
\NormalTok{X1 <-}\StringTok{ }\NormalTok{X.linha}\OperatorTok\NormalTok{X}
\KeywordTok{head.matrix}\NormalTok{(X1)}
\end{Highlighting}
\end{Shaded}

\begin{verbatim}
##             (Intercept) Claridade  Aroma  Corpo  Sabor Afinacao
## (Intercept)        38.0     35.10 184.20 178.00 181.20   161.70
## Claridade          35.1     32.99 170.45 163.25 166.97   149.98
## Aroma             184.2    170.45 936.24 880.95 908.67   789.78
## Corpo             178.0    163.25 880.95 858.92 869.05   760.86
## Sabor             181.2    166.97 908.67 869.05 903.14   776.10
## Afinacao          161.7    149.98 789.78 760.86 776.10   708.23
\end{verbatim}

\begin{itemize}
\tightlist
\item
  Calculando a inversa de \((X'X)^-1\):
\end{itemize}

\begin{Shaded}
\begin{Highlighting}[]
\NormalTok{X.inversa <-}\StringTok{ }\KeywordTok{solve}\NormalTok{(X1)}
\KeywordTok{head.matrix}\NormalTok{(X.inversa)}
\end{Highlighting}
\end{Shaded}

\begin{verbatim}
##             (Intercept)   Claridade        Aroma       Corpo       Sabor
## (Intercept)  3.68538435 -2.16087923  0.056193065 -0.29618518 -0.02052109
## Claridade   -2.16087923  2.22687731 -0.081481738  0.16333471  0.01106341
## Aroma        0.05619307 -0.08148174  0.054922315 -0.01420775 -0.03576244
## Corpo       -0.29618518  0.16333471 -0.014207755  0.08183244 -0.02875529
## Sabor       -0.02052109  0.01106341 -0.035762437 -0.02875529  0.06859675
## Afinacao    -0.10580781 -0.07494938 -0.002367866 -0.00752418 -0.00205549
##                 Afinacao
## (Intercept) -0.105807813
## Claridade   -0.074949380
## Aroma       -0.002367866
## Corpo       -0.007524180
## Sabor       -0.002055490
## Afinacao     0.054417686
\end{verbatim}

\begin{itemize}
\tightlist
\item
  Calculando a variância do modelo:
\end{itemize}

\begin{Shaded}
\begin{Highlighting}[]
\NormalTok{U.summary <-}\StringTok{ }\KeywordTok{summary}\NormalTok{(U)}
\NormalTok{variancia.modelo <-}\StringTok{ }\NormalTok{(U.summary}\OperatorTok{$}\NormalTok{sigma)}\OperatorTok{^}\DecValTok{2}
\NormalTok{variancia.modelo}
\end{Highlighting}
\end{Shaded}

\begin{verbatim}
## [1] 1.3515
\end{verbatim}

\begin{itemize}
\tightlist
\item
  Por fim, calculando a variância de \(\widehat{\beta_4}\). Observamos
  na matriz \emph{X.inversa} (calculada acima) que o valor da estimativa
  do mínimo quadrado de \(\widehat{\beta_4}\) (sabor) encontra-se na
  linha 5 e coluna 5.
\end{itemize}

\begin{Shaded}
\begin{Highlighting}[]
\NormalTok{beta4.variancia <-}\StringTok{ }\NormalTok{variancia.modelo }\OperatorTok{*}\StringTok{ }\NormalTok{X.inversa[}\DecValTok{5}\NormalTok{,}\DecValTok{5}\NormalTok{]}
\NormalTok{beta4.variancia}
\end{Highlighting}
\end{Shaded}

\begin{verbatim}
## [1] 0.09270852
\end{verbatim}

\end{document}
